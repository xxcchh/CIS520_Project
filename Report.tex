\documentclass[]{article}
\usepackage{amsfonts,amssymb,amsmath}
%\documentstyle[12pt,amsfonts]{article}
%\documentstyle{article}
\usepackage{biblatex}

\setlength{\topmargin}{-.5in}
\setlength{\oddsidemargin}{0 in}
\setlength{\evensidemargin}{0 in}
\setlength{\textwidth}{6.5truein}
\setlength{\textheight}{8.5truein}
\setcounter{MaxMatrixCols}{16}
%\input ../basicmath/basicmathmac.tex
%
%\input ../adgeomcs/lamacb.tex
\input ./mac-new.tex
\input ./mathmac-v2.tex
%\input ../adgeomcs/mac.tex
%\input ../adgeomcs/mathmac.tex

\def\fseq#1#2{(#1_{#2})_{#2\geq 1}}
\def\fsseq#1#2#3{(#1_{#3(#2)})_{#2\geq 1}}
\def\qleq{\sqsubseteq}

%
\begin{document}

\title{CIS520 Report: ML Crackers}   % type title between braces
\author{Francine Leech, Ziyin Qu, Chen Xiang}         % type author(s) between braces
\date{December 12, 2016}    % type date between braces
\maketitle

\section{Introduction}

The rise in web and mobile based social networking has opened a stream of continuous text data. These data reflect the sentiment of individuals and masses of people. Understanding the sentiment of users is relevant on the most basic level, understanding how people are responding to a stimuli, and extending to human machine interaction systems. The ambiguity of language and emotional expression creates an interesting machine learning problem.  We try to tackle one aspect of this problem by classifying tweets into two categories, joy and sadness. 

\section{Preliminary Methods}

When starting the project, we thought that we should first experiment with different classification methods on image data like train\_color, train\_img\_prob because they were smaller datasets. Using our inuition we made the assumption that the image data contained basic information about people's sentiments. For example, lighter and bright colors represent joy, darker colors represent sadness. \\\\

Because the image data alwats contains a lot of features, so the method we use for image data is PCA and Gaussian Mixture Model. With PCA, we can reduce the dimensions of features, with GMM, if it works, we may cluster the training data into two clusters, joy and sadness and thus get the trained classifier. \\\\

For the training, we tried three datasets for PCA and GMM, that is train\_color, train\_img\_prob and train\_cnn\_feat. However, the results are not satisfying. The table below demonstrates the trainning accuracies with PCA and GMM on the three datasets.

\[
\begin{tabular}{|c|c|c|c|}
\hline &train\_color & train\_img\_prob & train\_cnn\_feat\\
\hline PCA and GMM&59\%&58\%&59\%\\
\hline
\end{tabular}
\]

We did not use cross validation for tuning the parameters because the training process will take a large amount of time and instead we used 4000 data to train and 500 hold out to test. We trained the image data in this way. We trained two GMM respectively for joy and sadness and each GMM has 5 clusters. For the PCA, we tried different numbers principle components for the three dataset but the results did not improve much.\\\\

We think the reason why PCA and GMM did not work on image data may because the image data itself does not contain enough information for this method to get a classifier, unlike other classical clustering problems like human face recognition or male and female recognition. The accuracy is not enough to beat the baseline 1, so we have to try different methods.

Dimensionality Reduction

We tried to reduce the dimensionality of the $word_train$ data using Principal Component Analysis to determine if we could classify the data with a PC-ed version of the training data using a Gaussian Mixture Model. We used a 'holdout' of 20\% of the train data to use as testing because cross validation with PCA took too long.  

$$Graph with PCA/GMM error/accuracy$$

We found that even with 

KMeans 


\section{Main Methods}

Here we describe the combination of models we used in our model. 


\subsection{Naive Bayes}

For supervised learning, Naive Bayes method is a very good generative method on classifying texts, like spam classification problem. Actually the words\_train data set uses bag of words model where counts of words matters and position of words does not matter. There are lots of advantages for Naive Bayes model, it can be a dependable baselien for text classification, it trains fast. Although the assumption of Naive Bayes which is the Conditional Independence Assumption may not be true, it may still work well on text classification problem.\\\\

To use the Naive Bayes model on words\_train dataset, we use the built-in matlab function fitNaiveBayes. For trainning data, we sue 9-fold cross validation error to estimate the test error for Naive Bayes model, which means we randomly choose 4000 data to train and 500 data to test. For the built-in function fitNaiveBayes, there are some parameters for us to choose, for the distribution parameter, because we are using the bag-of-words model, we use the multinomial distribution in the fitNaiveBayes function. \\\\

The Naive Bayes model actually works really well. We got around 0.8 cross validation accuracy on trainning data. And we got 0.7962 accuracy for the test data. We successfully beat the baseline1 using a simple Naive Bayes model with multinomial distribution. \\\\

But there are still problems with the simple Naive Bayes model. For example, the dataset matrix for words is very sparse, and for each observation there are many  words did not show up. Naive Bayes model for text assumes that there is no information in words that are not observed and this may cause overfitting. We can solve this by smoothing the Naive Bayes model.

\subsection{GentleBoost}

The second main method we utilized was an ensemble method. 

Benefits of ensemble methods 



We used GentleBoost, a weak learning that was built by MATLAB under the fitensemble function. The method combines many weak learners into one high quality ensemble predictor. We chose this ensemble methods over the others offered by MATLAB, because it is preforms well with binary classification trees with many predictors (ensemble citation). \\

The input of the model was the $word_train$ data. We used a 10-fold cross validation method to observed how the model preformed, specified the use 300 learners, and the type of learner as 'tree'. The average cross validation error was 0.21. The algorithm classified joy and sadness well. \\

The method could have improved if we increased the number of learners, however it would have taken a very long time to train because the data is large. Initially we tried the method with the default number of learners, 100 trees, and found that the cross validation accuracy only improved slightly. This slight improvement with triple number of learners reveals that the data has some intricacies or patterns that the ensemble method cannot learn.   


\subsection{Support Vector Machine}

Support vector machines (SVMs) proved to the most promising method to classify the data. We used the MATLAB function, fitcsvm, to train an SVM model for binary classification on the on the $word_train$ data.

We tried a simple SVM by specifying a linear kernel, and had a  cross validation error was 0.2180. With a Gaussian or RBF kernel we had an error of 0.4373. 


After experimenting with a variety of kernels, we found that the linear preformed the best. fitsvm allows you to make an assumption about the fraction of outliers in the data. While we could have gone through the raw tweets and looked through the data, we decided to experiment with  10\%, 20\%, and 30\% and observed cross validation errors 0.2121, 0.2282, and 0.2131 respectively. Specifying the outlier percentage did not have an effect on our cross validation error, so we decided not to specify in our SVM final model. 

Lastly we optimized our SVM by using MATLABs built in method to optimize a cross-validated SVM using Bayes Optimization (citation). The method originates from The Elements of Statistical Learning, Hastie, Tibshirani, and Friedman (2009). Paraphrasing from the MATLAB documentation,  "the model begins with generating 10 base points for a "green" class, distributed as 2D independent normals with mean (1,0) and unite variance. It then generates 10 base points for a "class" that is also distributed as 2-D independent normals with mean (0,1) and unit variance. For each of the classes, it generate 100 random points by choosing a base point, b, of the respective color uniformly at random. It then generates an independent random point with 2-D normal distribution with mean b and variance I/5, where I is the 2-by-2 identity matrix. After 100 points for each of the colors has been generated, the point are classified using fitcsvm. The function bayesopt is used to optimize the parameters of the final SVM model with respect to cross validation." We submitted the method to the autograder, and it had an accuracy of 0.7991. The method was accurate enough to beat Baseline 1, but not Baseline 2. Similar to the other methods above, the optimized SVM may not preform well because the data was sparse and high dimensional, so the hyperplane could not separate data well. 

\section{Final Method}

Our final method utilizes sentiment analysis, the classification of text into categories of emotions. We used the vaderSentiment 2.4.1 package in Python. VADER is a lexicon and rule-based sentiment analysis tool that is specifically attuned to sentiments expressed in social media. The advantage of this analysis is that we can use this preliminary method to discern extremely positive and extremely negative tweets by assuming that there are no extremely positive words shown in negative sentences and vise versa. \\

\subsection{Sentiment Analysis 1}

In Python, we ran a sentimental analysis on each word present in the topwords\.csv. The input was each individual word in the list, and the output was the probability the word expresses a negative, positive, and neutral emotion. The output often looks like this, when type "funny" we can see \{'compound': 0.4404, 'neg': 0.0, 'neu': 0.0, 'pos': 1.0\}, which means it is an extremely postive word. \\


\subsection{Sentiment Analysis 2}

An issue we came across is from our first sentiment analysis is that we did not consider the raw tweets containing words that began with \#. These hashtags may represent the topic this sentence belongs to, some specific topics always express similar emotions, like \#family usually expresses a positive emotion. So instead of analyzing the sentiment of each word, we used sentence as the input and received the average emotion scores of each word.\\

We ran the sentimental analysis on each raw tweet using VADER package. For all the words that appeared in this sentence, we attached the resulting score to those words. For every word, an average emotion score is then based on all the raw tweets. \\ 

\subsection{Ensemble Method}




%Bar graph with the two errors 

%confusion matrix 


\section{Discussion}

Sentiment analysis is a difficult problem because human emotion is multifaceted and varies in intensity. In person, understanding the emotion a person is based on several parameters, the context of what they're saying, their words, tone, body language, and facial expression. Our classification problem is more difficult because we are given short tweets with a corresponding image. \\

One issue 
Issues with Lexicon
Developing the content of the lexicon is subjective. The use of language changes - how old is the lexicon being used? We don't know if the lexicon is comprehensive enough. Lexicon with emojis. 

What would our future method look like?
- Train classifier that train on emojis and hashtags since they are usually represent the topic of the tweet


Neutral tweets 


- Could have used deep learning 
	too long, overfit, dataset small
- articles 


\section{Works Cited}
\begin{thebibliography}{1}

%\bibitem{ensemble} Documentation. Ensemble Methods - MATLAB & Simulink. N.p., n.d. Web. 11 Dec. 2016. 

\end{thebibliography}

\end{document}
