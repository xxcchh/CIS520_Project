\documentclass[]{article}
\usepackage{amsfonts,amssymb,amsmath}
%\documentstyle[12pt,amsfonts]{article}
%\documentstyle{article}

\setlength{\topmargin}{-.5in}
\setlength{\oddsidemargin}{0 in}
\setlength{\evensidemargin}{0 in}
\setlength{\textwidth}{6.5truein}
\setlength{\textheight}{8.5truein}
\setcounter{MaxMatrixCols}{16}
%\input ../basicmath/basicmathmac.tex
%
%\input ../adgeomcs/lamacb.tex
\input ./mac-new.tex
\input ./mathmac-v2.tex
%\input ../adgeomcs/mac.tex
%\input ../adgeomcs/mathmac.tex

\def\fseq#1#2{(#1_{#2})_{#2\geq 1}}
\def\fsseq#1#2#3{(#1_{#3(#2)})_{#2\geq 1}}
\def\qleq{\sqsubseteq}

%
\begin{document}

\title{\TeX\ and \LaTeX}   % type title between braces
\author{Tom Scavo}         % type author(s) between braces
\date{October 27, 1995}    % type date between braces
\maketitle

\begin{abstract}
  A brief introduction to \TeX\ and \LaTeX
\end{abstract}

\chapter{\TeX}             % chapter 1
\section{Introduction}     % section 1.1
\subsection{History}       % subsection 1.1.1

\chapter{\LaTeX}           % chapter 2
\section{Introduction}     % section 2.1
\subsection{Usage}         % subsection 2.1.1

\begin{thebibliography}{9}
  % type bibliography here
\end{thebibliography}

\end{document}